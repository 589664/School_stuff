\section{Device Communication}
\label{sec:device_communication}

Implementation of network communication in ACD is made in RestClient class, as already mentioned with use of OkHttp library that is used to constructs the client, request, and response. This makes it possible for ACD issue requests on the cloud service. For this system there was implemented two methods. First one is doGetAccessCode() that issues HTTP GET request on cloud service to obtain current access code. Method first constructs a client by using OkHttpClient class, this client is later used to make a call and execute a request that results in a response from cloud service. Further method obtains code from response that contains a body. In order to obtain code from the body, method is using fromJson() method from Gson library. Method returns AccessCode object. 
\newline
The way GET request is constructed is with use of OkHttp library that contains Request class, where tha path of the request URL is provided. Together with URL we send path of request to achieve code from cloud service. As mentioned above there’s used "/accessdevice/code" path for that purpose. As well as get() method to specify that it’s a GET request and build() method to build the request. 
\newline\newline
When it comes to POST request, it is also implemented in RestClient class. POST request is some more complex in implementation rather than GET request. And is implemented in method doPostAcessEntry(String message) that suppose to send over the entry as message to cloud service storage (log). Same as previous method it is used client object of OkHttp library to make a call and execute the request. 

POST request is built in the same way as GET request beside that this time it’s used post() method to specify its POST request. This method requires RequestBody that again requires MediaType. MediaType describes the content type of an HTTP request or response body. While RequestBody that converts body of request into java object. For the rest of the implementation, it is the same as GET request method. Both of methods establishes a connection between cloud service and ACD. In that way that ACD can make appropriate HTTP GET and HTTP POST requests. 
