\newpage 
\section{Introduction}
\label{sec:Introduction}
The system that has been developed in this project consists of two parts. Part A and part B. In part A it has been created the hardware of the system, that represents ACD (access control device). The way it has been implemented, is that it’s running as a simulation of Ardruino board by using TinkerCAD website. That is providing tools for creation of a circuit though a website GUI, and running a simulation based on code provided by the user written in C++ programming language. Concept of ACD is that a motion sensor is getting triggered, the system goes over from locked to waiting mode where we can enter a code by using buttons. If the code is correct the system is set in unlocked mode.   

Part B is implementing a virtual part of ACD that uses HTTP – protocol to connect to a REST (Representational state transfer) based cloud service. Simulation of hardware ACD is later changed to use JavaFX GUI, since TinkerCAD could not provide the connection of circuit design to the internet.

System results in a IoT (internet of things) – cloud solution, that provides connection between ACD and cloud service. This makes it possible to track state change of system by storing locked/unlocked attempts in cloud. Change of access code is also getting tracked.


